\chapter*{Preface}
\addcontentsline{toc}{chapter}{Preface}

Learning how to program does not only involve an understanding of the vocabulary and the grammatical rules of a specific programming language, but also needs of a huge amount of concepts around the language and the computer like programming theory, the purpose of the language, the programming paradigm, software design, etc. One question that appears when someone has decided which language to learn is; how can I ask my computer to execute the code I am going to write?. The answer needs from the concepts of Compiler, Interpreter and maybe from the decision of using a Text Editor or an Integrated Development Environment (IDE).

Whether we start programming in a self-taught way or guided by a teacher, sometimes it is taken for granted that the basic concepts are known or they are treated really quickly. In addition, the issues related to the programming environment are usually underestimated since they are considered negligible compared to the programming language topics. However, the amount of time saved when the development environment is controlled is huge, specially when the project is comprised by a large number of files. We must not forget that a program like a web application, a numerical simulation or an Arduino project are not a unique file. They are usually composed by files, folders, projects, solutions and generally a strong organisation of all the elements. This guide is intended to help users to faster develop programs by using Visual Studio, one of the most powerful IDE.

This guide is specially aimed at the Aerospace Engineering students of the Technical University of Madrid who may start programming with different programming languages. We want them to start using Visual Studio quickly and without any complication. Furthermore, they become familiarised with an environment that could be really useful for their future career in case they continue in the world of software writing. 

For these purposes, the chapter 1 deals with the installation of Visual Studio in a Windows OS and some essential tools associated to the IDE. It also treats how to configure the environment in order to start using some really powerful functions. 

Chapters 2, 3, 4 and 5 cover different programming languages; Python, C++ and Arduino, Fortran and JavaScript with HTML and CSS. In all of the chapters, it is explained how to install all the components involved. Later a project is created for each language and a really simple program called ``Hello world'' is executed in order to validate the installation. In addition, in each language: configuration aspects, tips, common questions, how to manage packages, modules, libraries, projects and solutions, etc. are treated. By doing so, the student can reduce the number of hours dedicated to the setting up of the computer while familiarising with the environment. In the case of Python or Arduino, the essential management of packages and libraries respectively are treated. The Fortran section deals with concepts related to the configuration of the project, the use of graphic libraries and a pile of interesting questions. In the case of the web programming, some fundamental concepts of HTML/CSS and JavaScript are explained. All the guide is complemented with tips result of the dedication for a long time to coding in the Department of Applied Mathematics to Aerospace Engineering.

Finally, the chapter 6 covers the use of Git and GitHub, a Version Control Systems (VCS) compatible with Visual Studio, and once again, a powerful tool when managing codes. 

The experience of the students of Numerical Computation and Computer Science of the Aerospace Engineering degree of the Technical University of Madrid have made possible this guide since their comments and doubts have accurately contributed to the development of the content.

\begin{flushright}
    Miguel Ángel Rapado
    
    \vspace{-0.2cm}
    July 2019
\end{flushright}


