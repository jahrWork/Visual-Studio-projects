\usepackage[english]{babel} % English language/hyphenation


%___________________ FONTS ______________________________________________
\usepackage{amsfonts}
\usepackage{amssymb}
\usepackage{amsmath}
%\usepackage{txfonts}
\usepackage{mathrsfs}
\usepackage{mathdots}
\usepackage{lmodern}
%\usepackage[classicReIm]{kpfonts}
\usepackage{physics}    %  partial and total derivatives  

 \usepackage[utf8]{inputenc}
%___________ MATH THEOREMS
\usepackage{amsthm}
\usepackage{cancel}
\usepackage{mathtools}
\usepackage{mleftright}




% __________________________FANCY CHAPTERS _____________________________ 
%Options: Sonny, Lenny, Glenn, Conny, Rejne, Bjarne, Bjornstrup
%\usepackage[Sonny]{fncychap}
%\usepackage[Glenn]{fncychap} %1
%\usepackage[Conny]{fncychap}
%\usepackage[Rejne]{fncychap}
%\usepackage[Bjarne]{fncychap} %1
%\usepackage[Bjornstrup]{fncychap}
\usepackage[Lenny]{fncychap}

%_______________________ GLOSSARIES & NOMENCLATURE

%\usepackage{nomencl}

%\usepackage{glossaries}

%__________________________HEADERS & FOOTERS_____________________________
\usepackage{fancyhdr}


\pagestyle{fancy} % Enables the custom headers/footers
\lhead{\thepage}
\chead[]{}
\fancyhead[R]{\leftmark}

%To make the head rule
\usepackage{etoolbox}
\makeatletter
\patchcmd{\headrule}{\hrule}{\color{mygray}\hrule}{}{}
\makeatother


\renewcommand{\headrulewidth}{0.5pt} % No header rule
\renewcommand{\footrulewidth}{0.5pt} % Thin footer rule


%%PARA DOS CARAS
%\fancyhf{}             %allows additional selectors H (header) and F (footer)
\fancyhead[RO]{ \thepage}
\fancyhead[LE]{ \thepage}
%\fancyhead[LE,RO]{...}
\fancyhead[RE]{\sffamily\scshape\footnotesize\leftmark}
\fancyhead[LO]{\sffamily\scshape\footnotesize\rightmark}
\fancyfoot[CE,CO]{}
\fancyfoot[LE,RO]{}











%\usepackage{footmisc} %[norule]

%
%\renewcommand{\headrulewidth}{0.5pt} % No header rule
%\renewcommand{\footrulewidth}{0.5pt} % Thin footer rule
%
%\fancyhead[RO]{ \thepage}
%\fancyhead[LE]{ \thepage}
%%\fancyhead[LE,RO]{...}
%\fancyhead[RE]{\sffamily\scshape\footnotesize\leftmark}
%\fancyhead[LO]{\sffamily\scshape\footnotesize\rightmark}
%\fancyfoot[CE,CO]{}
%\fancyfoot[LE,RO]{}




%%PARA DOS CARAS
%\fancyhf{}             %allows additional selectors H (header) and F (footer)
%\pagestyle{fancy}
%\fancyhead[RO]{ \thepage}
%\fancyhead[LE]{ \thepage}
%%\fancyhead[LE,RO]{...}
%\fancyhead[RE]{\sffamily\scshape\footnotesize\leftmark}
%\fancyhead[LO]{\sffamily\scshape\footnotesize\rightmark}
%\fancyfoot[CE,CO]{}
%\fancyfoot[LE,RO]{}





%Allow us to make the rules larger
%
%\renewcommand{\headrulewidth}{0.5pt} % No header rule
%\renewcommand{\footrulewidth}{0.5pt} % Thin footer rule
% Headers - all currently empty
%\lhead{ }
%\chead{}
%\rhead{}
%% Footers
%\lfoot{}
%\cfoot{}
%%\rfoot{\footnotesize Page \thepage\ of \pageref{LastPage}} % "Page 1 of 2"
%\rfoot{\footnotesize \thepage\ }

%\usepackage{lettrine} % Package to accentuate the first letter of the text


%
%\renewcommand{\headrulewidth}{0.5pt} % No header rule
%\renewcommand{\footrulewidth}{0.5pt} % Thin footer rule
%
%
%%\renewcommand{\headrulewidth}{0.5pt} % No header rule
%%\renewcommand{\footrulewidth}{0.5pt} % Thin footer rule
%
%%%PARA DOS CARAS
%%\fancyhf{}                %allows additional selectors H (header) and F (footer)
%\fancyhead[RO]{ \thepage}
%\fancyhead[LE]{ \thepage}
%%\fancyhead[LE,RO]{...}
%\fancyhead[RE]{\sffamily\scshape\footnotesize\leftmark}
%\fancyhead[LO]{\sffamily\scshape\footnotesize\rightmark}
%\fancyfoot[CE,CO]{}
%\fancyfoot[LE,RO]{}
















%_________________ GRAPHS ______________________________________________
\usepackage{float} % Force image position
\usepackage{color}
\usepackage[dvipsnames]{xcolor}
\usepackage[font=small]{caption}
\usepackage{graphicx} % For pictures
\usepackage{epstopdf}
\usepackage{placeins}
\usepackage{tikz}
\usetikzlibrary{arrows,arrows.meta,positioning,shapes.geometric,shapes}
\usetikzlibrary{decorations.markings}

\usetikzlibrary{matrix}
\usepackage{pgfplots}
\usepackage{pgfplotstable}
\usepackage{pgf}
\usepackage{pgfcalendar}
%\usepackage{gnuplottex}
\usepgfplotslibrary{dateplot,statistics}
\usepgfplotslibrary{patchplots}
\usepgfplotslibrary{colormaps}
\usetikzlibrary{colorbrewer}
\usepgfplotslibrary{colorbrewer}
\usetikzlibrary{calc,intersections} 
\pgfplotsset{compat=1.12}
\usepackage{xifthen}
\usepackage{shellesc}
%\usetikzlibrary{external}
%\tikzexternalize[prefix=figures/]
%, shell escape=-enable-write18, shell escape=-enable-write18
%________________________ TABLES________________________________________
\usepackage{booktabs} % Horizontal rules in tables


%_____________________ TABLE OF CONTENTS _______________________________
\usepackage{makeidx} %Table of contents
%\PassOptionsToPackage{hyphens}{url}\usepackage{hyperref}
\PassOptionsToPackage{hyphens}{url}\usepackage[driverfallback=dvipdfmdvipdfm,colorlinks=true,linkcolor=black,breaklinks=true]{hyperref}  % Para incluir hipervinculos en el pdf
\hypersetup{citecolor=blue, urlcolor=blue}%,hyperfootnotes=false}       %Blue is different from blue, Blue is darker


\usepackage{enumitem} % Edit the itemizes
%\usepackage{titletoc,tocloft} % this two lines quit the indent that latex make between
% To change the format to the subsections
\usepackage{titlesec}
\usepackage{todonotes}


% ______________________EXTRA SPACE IN TABLES ______________________________
\usepackage{setspace}

 %_____________________________________ It is used to determine the number of pages in the document ____________________________(for "Page X of Total")
\usepackage{lastpage}



%\usepackage{extsizes} %14pt font size is not in the normal sizes of latex, we need this package


%_________________________The easylist package for numbered items____________________________________
%\usepackage[ampersand]{easylist}
\usepackage{subcaption}



%___________________ It enhances cross-reference ________________________________________
%\usepackage{cleveref}
%\usepackage{nameref}

%__________LUALATEX

%\usepackage{luatex85}



\usepackage{titling} % Allows custom title configuration
%\usepackage[a4,frame,center]{crop}






%\graphicspath{ {images/} }
%\newcommand{\folder}{./libraries}


\definecolor{DarkGoldenrod}{rgb}{0.8,0.6,0.1}
\definecolor{DarkRed}{rgb}{0.5,0.1,0.1}
\definecolor{DarkRedPart}{rgb}{0.3,0.1,0.2}

% Defines the gold horizontal rule around the title
%\newcommand{\HorRule}{\color{DarkGoldenrod} \rule{\linewidth}{1pt}}




\usepackage{titling} % Allows custom title configuration
% Defines the gold horizontal rule around the title
\newcommand{\HorRule}{\color{DarkGoldenrod} \rule{\linewidth}{1pt}}

\newcommand{\vect}[1]{\boldsymbol{#1}}




\definecolor{mygray}{gray}{0.8}
\definecolor{mygray2}{gray}{0.4}
\definecolor{mygreen}{rgb}{0.0, 0.42, 0.24}
\definecolor{myred}{rgb}{0.8, 0.0, 0.0}
\definecolor{myorange}{rgb}{0.91, 0.41, 0.17}
\definecolor{myblue}{rgb}{0.0, 0.44, 1.0}







%To make the head rule
\usepackage{etoolbox}
\makeatletter
\patchcmd{\headrule}{\hrule}{\color{mygray}\hrule}{}{}
\makeatother




%\setlength{\cftsecindent}{-0.8cm} % the section and the subsection in the table of contents
%\setlength{\cftsubsecindent}{-0.6cm} % the section and the subsection in the table of contents


% Replace the title of the table of contents 
\addto\captionsenglish{
    \renewcommand{\contentsname}%
    {}% Here it goes the new title
}

% Here we give the color to the subsections in the table of contents

%\makeatletter
%\let\stdl@subsection\l@subsection
%\renewcommand*{\l@subsection}[2]{%
%   \stdl@subsection{\textcolor{blue}{#1}}{\textcolor{black}{#2}}}
%\makeatother

% CHANGING THE FORMAT OF SUBSECTIONS AND SECTIONS

%\titleformat\subsection{}{}{0em}{ \bf \todo[inline, color=green!40]}[\vspace{0ex}]



%\titleformat\section{\centering \LARGE}{}{0em}{ \bf }[]

%\renewcommand{\thesubsection}{} %Subsection numbering
%\renewcommand{\thesection}{} %Section numbering



%\newcommand{\bluepageref}[1]{\textcolor{blue}{\pageref{#1}}}
\newcommand{\qnameref}[1]{``\nameref{#1}''}



% Comment the following to have chapters numbered without interruption (numbering through parts)
\makeatletter\@addtoreset{chapter}{part}\makeatother%

% \tableofcontents, without the title contents
%\makeatletter
%\@starttoc{toc}
%\makeatother
%



\newcommand{\remark}[1]{\textbf{``#1"}}





%\newcommand{\tit}{Raspberry Pi Server}
%\newcommand{\rhdr}{Developer Manual}


%\DeclareCaptionFormat{myformat}{#1#2#3\vspace{2ex}}
%\captionsetup{format=myformat}

%\captionsetup[lstlisting]{position=bottom,format=myformat}
%\usepackage{lstautodedent}

%commands

%\newcommand{\relpathto}[1]{./codes/#1}

%\renewcommand{\medskip}{\vspace{2ex}}

%\newcommand{\capof}[2]{\begingroup\captionof{#1}{#2}\endgroup}





%\usepackage{titling} % Allows custom title configuration
% Defines the gold horizontal rule around the title
%\newcommand{\HorRule}{\color{DarkGoldenrod} \rule{\linewidth}{1pt}}


%\graphicspath{ {images/} }



%\graphicspath{ {images/} }



% __________________________TEXT COLOR BOXES _____________________________ 
\usepackage{tcolorbox}      %Por alguna razón esto hay que declararlo después que scolor me parece, si no me salen muchos fallos. 
%\tcbuselibrary{listingsutf8} % o listings o minted, creo que para poner texto en latex y compilado al lado



%%%%%%%%%%%%%%%%%%%%%%%%%%%%%%%%%%%%%%%%%%%%%%%%%%%%%%%%%%%%%%%%%%%%%%%%%%%
%%                           TCOLORBOX BOXES                             %%
%%%%%%%%%%%%%%%%%%%%%%%%%%%%%%%%%%%%%%%%%%%%%%%%%%%%%%%%%%%%%%%%%%%%%%%%%%%
\newtcolorbox{IN}
{colback=blue!5!white,colframe=Blue!75!black,sharp corners=northwest, fonttitle=\bfseries, title=Important Notice}

\newtcolorbox{INWarning}
{colback=red!5!white,colframe=Red!75!black,sharp corners=northwest, fonttitle=\bfseries, title=Important Notice}

\newtcolorbox{mybox}[1]
{colback=blue!5!white,colframe=Blue!75!black,sharp corners=northwest, fonttitle=\bfseries, title=#1}

\newtcolorbox{ExampleCode}[1]
{colback=myred!5!white,colframe=myred!75!black,sharp corners=northeast, fonttitle=\bfseries, title=Example Code: #1}

\newtcolorbox{Example}[1]
{colback=myred!5!white,colframe=myred!75!black,sharp corners=northeast, fonttitle=\bfseries, title=Example: #1}


\usepackage[square,numbers]{natbib}


